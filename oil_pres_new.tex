%oil_pres_new.tex

\documentclass[12pt]{article}
\usepackage{setspace}
\usepackage{graphicx}
\usepackage{amsmath}
\usepackage{natbib} %for citet and citep
\usepackage{syntonly}
\usepackage{esdiff} %for writing partial derivatives
\usepackage{url} %for inserting urls
\usepackage{placeins} % for limitting floats
\usepackage{comment} %for excluding figures
% \excludecomment{figure} % for printing without figures
% \syntaxonly % for quickly checking document
%set document settings

\doublespacing % from package setspacs

% table font size
\let\oldtabular\tabular
\renewcommand{\tabular}{\scriptsize\oldtabular}

\title{The Effect of Oil Price on Field Production: Evidence from the Norwegian Continental Shelf}
\author{Johannes Mauritzen\\
		Department of Business and Management Science\\
        NHH Norwegian School of Economics\\
        Helleveien 30, 5045\\
        Bergen, Norway\\
        johannes.mauritzen@nhh.no\\
        \url{jmaurit.github.io}\\
		}
\date{\today}


\begin{document}
 \begin{spacing}{1} %sets spacing to single for title page
	\maketitle


\begin{abstract}
I use detailed field-level data on Norwegian off-shore oil field production and a semi-parametric additive model to control for the production profile of fields to estimate the effect of oil prices on production.  I find no significant evidence of a concurrent reaction of field production to oil prices, though a slight lagged effect is found of the magnitude of approximately 2 to 4\% for a 10 dollar per barrel increase in the real price of oil.  Most of this effect appears to come in the planning phase of a field's development.\\
Keywords: Oil production,Oil Prices, Norway, Semiparametric.
\end{abstract}

\thanks{I would like to thank Klaus Mohn, Jonas Andersson, Sturla Kvamsdal, Harrison Fell, R\"ognvaldur Hannesson and Henrik Horn for valuable discussion and comments.}
% JEL Codes: Q4, L71
 \end{spacing}

\section{Introduction}

For most of the last century, crude oil has been the world’s single most important and valuable fuel source.\footnote{Oil is the world's largest single energy source, consisting of approximately 33\% of total energy consumption in 2013 as well as the most valuable in terms of market price per unit of energy \citep{british_petroleum_statistical_2013}} Naturally, questions of how the oil price affects the world economy as well as how oil production reacts to the oil price have been fundamental topics in economics. 

However a significant gap exists in the literature.  While numerous studies have taken up the issue of how the price of oil affects searching for new fields as well as total oil production at both the regional and global level, few studies exist of the effects of oil prices on oil production at the field level.  The studies that do use field- and well-level data, such as \citet{rao_taxation_2010}, often use data from on-shore installations.  However an increasing amount of the world's oil production comes from hard-to-reach off-shore fields.  Production from challenging off-shore environments is substantially different in character than from on-shore installations.

Moreover, studies within the economics literature fail to take into account how oil prices have varying effects on the different stages of field development and production.  The planning, build-out and depletion phase of a field are conceptually unique and how oil companies operating off-shore respond to oil prices can be expected to vary substantially.

The effect of oil prices on producing fields is an important topic in understanding the mechanisms of how total oil supply reacts in response to price. The response of oil production in existing fields is especially important now as many of the major oil- producing areas, like the Norwegian Continental Shelf are mature and an increasing share of investments will be directed at capacity enhancement in producing fields rather than new field developments. How production from these fields will respond to changes in oil prices has major implications for the oil industry, the state finances of oil-producing countries and long-run oil price formation.

The question of the effect of oil prices on production and the more general question of optimal oil extraction has spawned a large theoretical literature dating back to the seminal work of \citet{hotelling_economics_1931}. \citet{krautkraemer_nonrenewable_1998} provides a good overview. At a basic level a central idea of much of this theoretical work is that with a non-renewable resource, production is a decision that involves a significant opportunity cost: more production in the current period means less production in future periods.  Within this frameworks, prices and expectations of prices become important variables in the production decision. A simple Hotelling model suggests that a producer would immediately change their production in response to a change in oil price in order to dynamically optimize the total extraction.

But in practice, the question is not as clear cut.  Production in the Norwegian Continental Shelf - as well as most other offshore production areas - has extremely high fixed and operating costs.  Keeping spare capacity available in order to adjust to changing oil prices might simply be too expensive.  Instead, producers may find it more beneficial to use storage and financial instruments in order to hedge short-term price movements. Higher oil prices may however still lead a producer to invest in increased capacity.  Since lag times are significant in the off-shore sector, we would then expect to see a multi-year lagged effect of prices on production.  

However even with the question of lagged production and investment, some ambiguity exists.  \citet{mohn_efforts_2008} suggests and finds evidence for the idea that in periods of high oil prices off-shore producers will invest more in risky wild-cat drilling in search of new fields, but concentrate investments in lower-risk ventures, like expanding production in existing fields, when prices are low.  If this effect were to dominate, then it may even be plausible that production in existing fields reacts \emph{negatively} to increases in price. 

%Implications for theories and explanations for understanding of the price mechanism and %the role of speculation.  Hamilton block quote.  

For such a prominent subject, the lack of research on the role of price in oil field production is striking. Two main factors likely contribute to the limited literature - the availability of data and the non-linear time profile of field production.  Large private oil companies, notably the “super majors” and state-owned oil companies have historically accounted for the vast majority of oil production and reserves. \footnote{See for example the economist article titled Supermajordämmerung from August 3rd, 2013: \url{http://www.economist.com/news/briefing/21582522-day-huge-integrated-international-oil-company-drawing}} These entities tend to consider field-level data as either company or state secrets.   

However, over the last few years, a movement towards making the petroleum and other extractive industries more transparent has taken form.  The Norwegian government has been on the forefront of this movement and increasingly committed itself to transparency in the petroleum sector. \footnote{see http://www.regjeringen.no/en/sub/eiti---extractive-industries-tranparency/about-eiti.html?id=633586} Over the last several years, detailed data on most aspects of the country\’s oil industry has become openly available.  

In this article, I use historical production data from all 77 currently or formerly oil-producing fields on the Norwegian continental shelf in order to estimate the effect that prices have on oil production.  

By looking only at the effect of price on fields that currently or previously have produced oil I am limiting the scope of this article.  The effect of oil prices on total production over an extended period of time is due not just to reactions in production in existing fields but also increased searching for new fields.  In fact, an implication of this work is that much of the total production response from higher oil prices is likely from increased searching as well as production from previously un-economic fields.

The main finding in this article is that oil production at the field level has no significant reaction to concurrent changes in the oil price, where concurrent is broadly defined as within the first three years.  A slight effect can be detected at a lag of between 4 and 8 years, with a magnitude of about a 2 to 4\% increase in yearly production for a 10 dollar increase in the price of oil.  This effect is somewhat greater and with less of a lag in large fields compared to small fields.  Price appears to have the most significant affect during the planning stage - before production begins in a field.  In the depletion phase of production, price is found to have little to no significant effect.  

The main methodological problem is the non-linear production profile of oil fields.  Once full-scale extraction is started in an oil field, pressure in wells will start declining. Compensating measures such as gas and water injection have been successful, but their impact is temporary. In turn production rates will therefore inevitably decline.

More so, oil field production is correlated across fields - that is, increases and decreases in production in fields are not randomly distributed across time.  Instead, as figure \ref{top10_production} shows with the production profile of the 10 largest Norwegian oil fields, production tends to be correlated across fields.  The result is a total production curve that is bell-shaped over time as in figure \ref{oil_decline}.  Since the oil price series is autocorrelated and non-stationary, failure to properly account for the production profile will lead to spurious estimation of the price terms in a regression.

The direction of this bias can be gleaned in figure \ref{oil_decline}.  High oil prices were present at periods of relatively low production in the late 1970s and early 1980s as well as over the last 10 years, however real prices reached some of their lowest levels at the same time as the top of production around the year 2000. These oil price dynamics almost certainly affected investment decisions in the industry and in turn total production (\citep{osmundsen_is_2007}, \citep{aune_financial_2010}). However the inverse contemporaneous relationship at the field level is entirely coincidental, but will heavily bias the estimation of the effect of price on production if field production profiles are not properly accounted for. 

\begin{figure}
	\includegraphics[width=1\textwidth]{figures/top10_production_print.png}
	\caption{The production profile of the 10 largest oil fields on the Norwegian Continental Shelf.  Production tends to be correlated across fields}
	\label{top10_production}	
	\end{figure}

\begin{figure}
	\includegraphics[width=1\textwidth]{figures/oil_decline_print.png}
	\caption{The production profile for the entire Norwegian Continental Shelf is bell-shaped, reflecting the correlated production profiles of the fields.  With oil prices that are autocorrelated, }
	\label{oil_decline}
\end{figure}

As a solution I use a semi-parametric model within the Generalized Additive Model frameworks of \cite{hastie_generalized_1990}.  Here I use a two-dimensional smoothed spline function to account for the general non-parametric shape of the production profile while allowing price to enter the equation linearly.  The coefficient of price can then be interpreted as the average effect of price on production over the entire production profile.

\FloatBarrier
\section{The effect of oil price on production: theory, simulations and empirics}

For modeling aggregate oil production, shape-fitting models, notably \citet{hubbert_energy_1962} and more recently advocated by \citet{deffeyes_hubberts_2001}, have had some success in estimating the timing of peak production at the regional and national level, but they tend to seriously underestimate the total recoverable resource of oil-producing regions and the models can be shown to be fundamentally misspecified \citep{boyce_prediction_2013}.   Simulation type studies where aggregate oil production is modeled through an often complex combination of physical and economic processes also exist in the geo-engineering literature, but their usefulness tends to be weighed down by their complexity as they require quite detailed data and specific assumptions about functional form that can be difficult to justify \citet{brandt_review_2010}.

More important to this paper are empirical estimates of the effect of price on production. Several econometric papers seek to answer the question of how aggregated oil supply is affected by oil prices.  \citet{farzin_impact_2001} attempts to estimate an elasticity for the effect on added reserves of increased oil prices and finds a small though statistically significant effect.  \citet{ramcharran_oil_2002} estimates a supply function for the total supply of oil from several OPEC countries based on data from 1973 to 1997.  The author finds a negative price elasticity for several of the countries, and interprets this as evidence of producers targeting revenue.  However since the author does not take into account the production profile of oil fields and the spurious correlation that can arise with autocorrelated prices, these estimates come under considerable doubt.  

the effect of oil-price uncertainty on drilling and exploration has also been explored.  \citet{kellogg_effect_2014} finds that oil exploration firms in Texas do approximately respond as real-option theory would predict when it comes to the timing of drilling.  A model and test using data from North Sea producers on the UK continental shelf by \citet{hurn_geology_1994}, on the other hand, fails to find evidence that the variance in the oil price affects the timing of oil field development.  I do not attempt to directly model uncertainty, however given that the investments needed to increase oil production in an existing field are to a certain extent irreversible and that oil prices are highly volatile, the results can and probably should be interpreted with the real options framework in mind.  

Only a few papers utilize field-level data.  \citet{black_is_1998} tests the relevance of “nesting” a structural empirical model of profit maximization that takes into account oil prices into a typical geo-engineering model of oil field production.  They find strong evidence that taking into account profit maximization, and implicitly price, substantially improves the fit compared to a purely geo-engineering type model.   The limitation of their methodology is that they are only able to test whether including economic factors like price affects the fit of the model but are not able to give an estimate of the effect.  Methodologically, the paper also relies heavily on assumptions about the functional form of both the geo-engineering aspects of the oil producer as well as their profit-maximization.  By taking a more flexible, semi-parametric approach to estimating the effects of oil field production profile, this paper avoids problems with overly restrictive assumptions. 

\citet{rao_taxation_2010} uses data on land based oil wells in California to estimate the effects of tax changes and price controls.  The author finds that short-term tax changes caused small, but significant retiming of production from oil wells.  These findings are over-all consistent with the findings of this paper, though the author finds concurrent effects of taxation on production while I do not find any concurrent effect of prices on production.  This difference is most likely due to the significant differences in cost and complexity between operating offshore and onshore.  How producers react to a short-term tax change as opposed to a change in price is likely also different.  More so, the author does not consider differences between the different stages of production.   

Studies using detailed Norwegian data on offshore activity also exist, though the focus has mainly been on exploration and drilling.  \citet{mohn_exploration_2008} finds that long-term changes in the oil-price has a strong effect on exploratory drilling though little effect is measured from short-term changes in the oil price.  \citet{osmundsen_exploration_2010} analyses drilling productivity over time on the Norwegian Continental Shelf while \citet{mohn_efforts_2008} finds that higher oil prices leads to higher reserves and as well as that oil prices affect producer risk-preferences - with higher prices leading to lower success rates but larger discovery size.  


\section{Oil production on the Norwegian Continental Shelf}


The first commercial oil well in Norwegian continental waters was discovered in December of 1969 in what became the Ekofisk oil field, the largest Norwegian oil field by estimated recoverable reserves.  As figure \ref{north_sea_reserves} shows, most of the largest fields in the North Sea were found relatively early on while more recent finds have tended to be smaller - a pattern typical of oil producing areas called creaming.  A major exception to this trend was the recent find of the Johan Sverdrup field which is estimated to have approximately 300 million Standard Cubic Meters (SM3) of recoverable oil. \footnote{The Johan Sverdrup field is estimated to begin producing oil in 2019 and so is not present in the data set used for the analysis.}  

\begin{figure}
\includegraphics[width=1.2\textwidth]{figures/north_sea_reserves_print.png}
\caption{Oil fields in the Norwegian territorial North Sea.  The largest oil fields tended to be discovered earliest, while newer finds tend to be smaller.  An exception is the large Johan Sverdrup field, which is expected to begin producing in 2019.}
\label{north_sea_reserves}
\end{figure}

Exploration in the Norwegian Sea was opened in the early 1980’s and the first commercial field started production in 1981.  While several mid-sized fields have been discovered, the Norwegian Sea has generally disappointed in terms of commercial oil finds and most finds have been relatively small (\ref{norwegian_sea_reserves}).  

\begin{figure}
\includegraphics[width=1.2\textwidth]{figures/norwegian_sea_reserves_print.png}
\caption{Oil fields in the Norwegian Sea.  Production from the Norwegian Sea has generally been dissapointing compared with expectations when the area was opened to exploration in the early 1980's.}
\label{norwegian_sea_reserves}
\end{figure}

Norwegian waters in the Barents Sea have also been open to exploration since the 1980s, however up until recently only a few, small finds were made and none came into commercial production.  However recently several significant oil and gas finds have been made in the Barents Sea - notably the oil field Goliat and the large gas find Sn\o hvit, which are both currently under development but not yet producing.  The agreement between Norway and Russia in June 2011 settling a long-running dispute over the maritime delimitation has also given a boost to new exploration in the region.  

Profits from oil and gas production in Norway are subject to a resource tax of 50 \% on top of the ordinary income tax of 28 \%, thus income from petroleum production is taxed at a total marginal tax rate of 78 \%.  The central government also receives revenues through ownership stakes in companies, notably Statoil, where the state is the majority stakeholder.  The over-all tax rate has been fairly constant through the history of Norwegian oil production, however several important changes related to the tax code have occurred.  In 1991 a C02 tax was introduced, and in the year 2012 the tax was doubled.  But this was mainly levied on the use and import of petroleum products.  In the offshore sector it was levied on the burning of oil and gas and thus the main effect was on the practice of flaring natural gas that could not be transported and sold commercially.   

More importantly to the offshore sector were accounting changes that were implemented in 2002 and 2005 that were meant to encourage new entrants by allowing losses to be carried forward for tax purposes and by introducing a rebate on the tax-value of losses associated with searching and drilling.  In general though, these rules mainly affected searching and discoveries of new fields rather than production from existing fields and I do not control for the tax changes in my model.  More information on taxation and revenues from the offshore sector can be found at the website of the Norwegian Ministry of Finance. \footnote{\url{http://www.regjeringen.no/en/dep/fin/Selected-topics/taxes-and-duties/bedriftsbeskatning/Taxation-of-petroleum-activities.html?id=417318}}

Rights to explore and eventually produce on the Norwegian Continental Shelf are based on a system where the government announces geographic blocks that will be opened to oil exploration and production subject to production licenses.  Production licenses are initially granted for between 4-6 years subject to requirements that firms are actively searching in their awarded blocks.  If oil or gas deposits are proven then the production license can be extended for up to 30 years.  In general, the frameworks are fair, predictable and stable for companies who find commercially extractable oil deposits and regulatory interference is unlikely to be the cause of any observed changes in production from existing oil fields.  For more information see the website of the Norwegian Petroleum Directorate. \footnote{\url{http://www.npd.no/en/Topics/Production-licences/Theme-articles/Production-licence--licence-to-explore-discover-and-produce-/}}

\section{Norwegian oil field production data}
Production data of Norwegian oil fields is obtained from the website of the Norwegian Petroleum Directorate. \footnote{http://factpages.npd.no/}  Production data is available at a monthly frequency for all fields, though I choose to aggregate up to yearly values both to smooth over seasonality as well as short-term volatility of output due to factors such as weather or technical issues that are not relevant for this article.  

In addition to data on field-level production, I also make use of data on estimated total recoverable reserves. The use of this variable is complicated as it is an estimate subject to a large amount of uncertainty, especially in relatively young fields.  However, the methodologies used to estimate the total recoverable resource of a field are constantly evolving and it is a fair assumption that any consistent bias of the estimates are observed in older fields and corrected for in estimates for newer fields.  I can then assume that existing errors are random and will not significantly bias the estimates.  

Moreover, the estimate is likely endogenous, in the sense that it is also effected by prices.  However since I use the variable as a control variable and not for the purpose of estimating a parameter with a causal interpretation, this should not significantly affect the validity of the results.  

I use yearly data from the US Energy Information Agency on the real price of Brent-traded oil in 2010 dollars.The Brent benchmark oil price is likely the best oil price measure for Norwegian production as it is based on light sweet crude oil sourced from the North Sea.  

An argument can be made that expectations of future oil prices can be equally if not even more important for production decisions as the current oil price.  Forecasts for future oil prices are available from, among others, the International Energy Agency, but these have tended to be notoriously inaccurate and it is unlikely oil companies use these projections for their investment decisions.  

On the other hand, given the size and liquidity of oil spot markets, it is a fair assumption that the current oil prices do a good job of incorporating much of the available information about crude oil markets and that future price movements are generally difficult to predict \citep{hamilton_understanding_2008}.  An active futures market does exist, but several studies have found that current oil prices are in general better than prices on futures contracts at predicting future oil prices \citep{alquist_what_2010, chinn_predictive_2005}.  \citet{mohn_investment_2008} as well as \citet{pesaran_econometric_1990} and \citet{farzin_impact_2001} find evidence for adaptive expectations, where expectations of future prices is based on a weighted average of current and past prices.  I take account of this by including several years of price lags in my regression equations.  

A cleaned data set and the full code for the analysis are available upon request. I use the R statistical programming package for all the analysis in this article \citep{r_core_team_r:_2013}.  I use the R packages ggplot2 and ggmap for plotting \citep{wickham_ggplot2:_2009, kahle_ggmap:_2013}, plyr for data manipulation and cleaning \citep{wickham_split-apply-combine_2011}, texreg for table formatting \citep{leifeld_texreg:_2013} and mgcv for implementation of the Generalized Additive Models \citep{wood_fast_2011}.
