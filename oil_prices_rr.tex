\documentclass[11pt]{article}
\usepackage[margin=1in]{geometry}
\usepackage{setspace}
\usepackage{graphicx}
\usepackage{amsmath}
\usepackage{natbib} %for citet and citep
\usepackage{syntonly}
\usepackage{esdiff} %for writing partial derivatives
\usepackage{url} %for inserting urls
\usepackage{placeins} % for limitting floats
\usepackage{comment} %for excluding figures
% \excludecomment{figure} % for printing without figures
% \syntaxonly % for quickly checking document
%set document settings

\doublespacing % from package setspacs

\title{The Effect of Oil Price on Field Production: Evidence from the Norwegian Continental Shelf}
\author{Johannes Mauritzen\\
		Department of Business and Management Science\\
        NHH Norwegian School of Economics\\
        Helleveien 30, 5045\\
        Bergen, Norway\\
        johannes.mauritzen@nhh.no\\
        \url{jmaurit.github.io}\\
		}
\date{\today}


\begin{document}
 \begin{spacing}{1} %sets spacing to single for title page
	\maketitle


\begin{abstract}
I use detailed field-level data on Norwegian off-shore oil field production and a semi-parametric additive model to control for the production profile of fields to estimate the effect of oil prices on production.  I find no significant evidence of a concurrent reaction of field production to oil prices, though a slight lagged effect is found of the magnitude of approximately 3 to 5\% for a 10 dollar per barrel increase in the real price of oil.  Most of this effect appears to come in the build-out or planning phase of a field's development.\\
Keywords: Oil production,Oil Prices, Norway, Generalized Additive Model.
\end{abstract}

\thanks{I would like to thank Klaus Mohn, Jonas Andersson, Sturla Kvamsdal, Harrison Fell, R\"ognvaldur Hannesson and Henrik Horn for valuable discussion and comments.}
% JEL Codes: Q4, L71
 \end{spacing}

\section{Introduction}

For most of the last century, crude oil has been the world’s single most important and valuable fuel source.\footnote{Oil is the world's largest single energy source, consisting of approximately 33\% of total energy consumption in 2013 as well as the most valuable in terms of market price per unit of energy \citep{british_petroleum_statistical_2013}} Naturally, questions of how the oil price affects the world economy as well as how oil production reacts to the oil price have been fundamental topics in economics. Recent volatility in the price of oil has made this an especially relevant topic. 

However a significant gap exists in the literature.  While numerous studies have taken up the issue of how the price of oil affects searching for new fields as well as total oil production at both the regional and global level, few studies exist of the effects of oil prices on oil production at the field level.  The studies that do use field- and well-level data, such as \citet{rao_taxation_2010}, often use data from on-shore installations.  However an increasing amount of the world's oil production comes from hard-to-reach off-shore fields.  Production from challenging off-shore environments is substantially different in character than from on-shore installations.

Moreover, studies within the economics literature fail to take into account how oil prices have varying effects on the different stages of field development and production.  The planning, build-out and depletion phase of a field are conceptually unique and how oil companies operating off-shore respond to oil prices can be expected to vary substantially.

The effect of oil prices on producing fields is an important topic in understanding the mechanisms of how total oil supply reacts in response to price. The response of oil production in existing fields is especially important now as many of the major oil- producing areas, like the Norwegian Continental Shelf are mature and an increasing share of investments will be directed at capacity enhancement in producing fields rather than new field developments. How production from these fields will respond to changes in oil prices has major implications for the oil industry, the state finances of oil-producing countries and long-run oil price formation.

The question of the effect of oil prices on production and the more general question of optimal oil extraction has spawned a large theoretical literature dating back to the seminal work of \citet{hotelling_economics_1931}. \citet{krautkraemer_nonrenewable_1998} provides a good overview. At a basic level a central idea of much of this theoretical work is that with a non-renewable resource, production is a decision that involves a significant opportunity cost: more production in the current period means less production in future periods.  Within this frameworks, prices and expectations of prices become important variables in the production decision. A simple Hotelling model suggests that a producer would immediately change their production in response to a change in oil price in order to dynamically optimize the total extraction.

But in practice, the question is not as clear cut.  Production in the Norwegian Continental Shelf - as well as most other offshore production areas - has extremely high fixed and operating costs.  Keeping spare capacity available in order to adjust to changing oil prices might simply be too expensive.  Instead, producers may find it more beneficial to use storage and financial instruments in order to hedge short-term price movements. Higher oil prices may however still lead a producer to invest in increased capacity.  Since lag times are significant in the off-shore sector, we would then expect to see a multi-year lagged effect of prices on production.  

However even with the question of lagged production and investment, some ambiguity exists.  \citet{mohn_efforts_2008} suggests and finds evidence for the idea that in periods of high oil prices off-shore producers will invest more in risky wild-cat drilling in search of new fields, but concentrate investments in lower-risk ventures, like expanding production in existing fields, when prices are low.  If this effect were to dominate, then it may even be plausible that production in existing fields reacts \emph{negatively} to increases in price. 

%Implications for theories and explanations for understanding of the price mechanism and %the role of speculation.  Hamilton block quote.  

For such a prominent subject, the lack of research on the role of price in oil field production is striking. Two main factors likely contribute to the limited literature - the availability of data and the non-linear time profile of field production.  Large private oil companies, notably the “super majors” and state-owned oil companies have historically accounted for the vast majority of oil production and reserves. \footnote{See for example the economist article titled Supermajordämmerung from August 3rd, 2013: \url{http://www.economist.com/news/briefing/21582522-day-huge-integrated-international-oil-company-drawing}} These entities tend to consider field-level data as either company or state secrets.   

However, over the last few years, a movement towards making the petroleum and other extractive industries more transparent has taken form.  The Norwegian government has been on the forefront of this movement and increasingly committed itself to transparency in the petroleum sector. \footnote{see http://www.regjeringen.no/en/sub/eiti---extractive-industries-tranparency/about-eiti.html?id=633586} Over the last several years, detailed data on most aspects of the country\’s oil industry has become openly available.  

In this article, I use historical production data from all 77 currently or formerly oil-producing fields on the Norwegian continental shelf in order to estimate the effect that prices have on oil production.  

By looking only at the effect of price on fields that currently or previously have produced oil I am limiting the scope of this article.  The effect of oil prices on total production over an extended period of time is due not just to reactions in production in existing fields but also increased searching for new fields.  In fact, an implication of this work is that much of the total production response from higher oil prices is likely from increased searching as well as production from previously un-economic fields.

The main finding in this article is that oil production at the field level has no significant reaction to concurrent changes in the oil price.  A slight effect can be detected at a lag of between 1 and 4 years, with a magnitude of about a 2 to 5\% increase in yearly production for a 10 dollar increase in the price of oil.  

% Price appears to have the most significant affect during the planning and build-out stage.  In the depletion phase of production, price is found to have little to no significant effect.  

The main methodological problem is the non-linear production profile of oil fields.  Once full-scale extraction is started in an oil field, pressure in wells will start declining. Compensating measures such as gas and water injection have been successful, but their impact is temporary. In turn production rates will therefore inevitably decline.

More so, oil field production is correlated across fields - that is, increases and decreases in production in fields are not randomly distributed across time.  Instead, as the top panel of figure \ref{oil_decline} shows with the production profile of the 10 largest Norwegian oil fields, production tends to be correlated across fields.  The result is a total production curve that is bell-shaped over time as in the lower panel of figure \ref{oil_decline}.  Since the oil price series is autocorrelated and non-stationary, failure to properly account for the production profile will lead to spurious estimation of the price terms in a regression.

\begin{figure}
	\includegraphics[width=1\textwidth]{figures/oil_decline.png}
	\caption{The panel shows the production from the largest 10 fields, which are temporally correlated with each other.  The total production over time is bell-shaped, as shown by the bottom panel.}
	\label{oil_decline}
\end{figure}

The direction of this bias can be gleaned in figure \ref{oil_decline}.  High oil prices were present at periods of relatively low production in the late 1970s and early 1980s as well as over the last 10 years, however real prices reached some of their lowest levels at the same time as the top of production around the year 2000. These oil price dynamics almost certainly affected investment decisions in the industry and in turn total production (\citep{osmundsen_is_2007}, \citep{aune_financial_2010}). However the inverse contemporaneous relationship at the field level is entirely coincidental, but will heavily bias the estimation of the effect of price on production if field production profiles are not properly accounted for. 



As a solution I use a semi-parametric model within the Generalized Additive Model frameworks of \cite{hastie_generalized_1990}.  Here I use both a cubic spline and a two-dimensional thin-plate spline to account for the general non-parametric shape of the production profile and allowing price to enter the equation linearly while also attempting to control for the effects of varying production costs and the effects of technological change and other time-varying effects.  

The results suggest that little to no concurrent effect of oil prices on production exists.  However modest lagged effects of between 1 to 4 years exist.These results are in line with an understanding of the industry as being highly capital intensive with high operating costs.  Maintaining spare capacity in order to adjust to changing oil prices is likely prohibitively inexpensive.  Instead, production is increased through investment, which may take up to several years to show an effect.

\FloatBarrier
\section{The effect of oil price on production: theory, simulations and empirics}

For modeling aggregate oil production, shape-fitting models, notably \citet{hubbert_energy_1962} and more recently advocated by \citet{deffeyes_hubberts_2001}, have had some success in estimating the timing of peak production at the regional and national level, but they tend to seriously underestimate the total recoverable resource of oil-producing regions and the models can be shown to be fundamentally misspecified \citep{boyce_prediction_2013}.   Simulation type studies where aggregate oil production is modeled through an often complex combination of physical and economic processes also exist in the geo-engineering literature, but their usefulness tends to be weighed down by their complexity as they require quite detailed data and specific assumptions about functional form that can be difficult to justify \citep{brandt_review_2010}.

More important to this paper are empirical estimates of the effect of price on production. Several econometric papers seek to answer the question of how aggregated oil supply is affected by oil prices.  \citet{farzin_impact_2001} attempts to estimate an elasticity for the effect on added reserves of increased oil prices and finds a small though statistically significant effect.  \citet{ramcharran_oil_2002} estimates a supply function for the total supply of oil from several OPEC countries based on data from 1973 to 1997.  The author finds a negative price elasticity for several of the countries, and interprets this as evidence of producers targeting revenue.  However since the author does not take into account the production profile of oil fields and the spurious correlation that can arise with autocorrelated prices, these estimates come under considerable doubt.  

the effect of oil-price uncertainty on drilling and exploration has also been explored.  \citet{kellogg_effect_2014} finds that oil exploration firms in Texas do approximately respond as real-option theory would predict when it comes to the timing of drilling.  A model and test using data from North Sea producers on the UK continental shelf by \citet{hurn_geology_1994}, on the other hand, fails to find evidence that the variance in the oil price affects the timing of oil field development.  

I do not attempt to directly model uncertainty, however given that the investments needed to increase oil production in an existing field are to a certain extent irreversible and that oil prices are highly volatile, the results can and probably should be interpreted with the real options framework in mind.  

Only a few papers utilize field-level data.  \citet{black_is_1998} tests the relevance of “nesting” a structural empirical model of profit maximization that takes into account oil prices into a typical geo-engineering model of oil field production.  They find strong evidence that taking into account profit maximization - and implicitly price - substantially improves the fit compared to a purely geo-engineering type model.   The limitation of their methodology is that they are only able to test whether including economic factors like price affects the fit of the model but are not able to give an estimate of the effect.  Methodologically, the paper also relies heavily on assumptions about the functional form of both the geo-engineering aspects of the oil producer as well as their profit-maximization.  By taking a more flexible, semi-parametric approach to estimating the effects of oil field production profile, this paper avoids problems with overly restrictive assumptions. 

\citet{rao_taxation_2010} uses data on land-based oil wells in California to estimate the effects of tax changes and price controls.  The author finds that short-term tax changes caused small, but significant retiming of production from oil wells.  These findings are over-all consistent with the findings of this paper, though the author finds concurrent effects of taxation on production while I do not find any concurrent effect of prices on production. The significant differences in cost and complexity between operating offshore and onshore likely account for the different results.  How producers react to a short-term tax change as opposed to a change in price is likely also different.

Studies using detailed Norwegian data on offshore activity also exist, though the focus has mainly been on exploration and drilling.  \citet{mohn_exploration_2008} finds that long-term changes in the oil-price has a strong effect on exploratory drilling though little effect is measured from short-term changes in the oil price.  \citet{osmundsen_exploration_2010} analyses drilling productivity over time on the Norwegian Continental Shelf while \citet{mohn_efforts_2008} finds that higher oil prices leads to higher reserves as well as that oil prices affect producer risk-preferences - with higher prices leading to lower success rates but larger discovery size.  


\section{Oil production on the Norwegian Continental Shelf}

The first commercial oil well in Norwegian continental waters was discovered in December of 1969 in what became the Ekofisk oil field, the largest Norwegian oil field by estimated recoverable reserves.  Most of the largest fields in the North Sea were found relatively early on while more recent finds have tended to be smaller - a pattern typical of oil producing areas called creaming.  A major exception to this trend was the recent find of the Johan Sverdrup field which is estimated to have approximately 300 million Standard Cubic Meters (SM3) of recoverable oil. \footnote{The Johan Sverdrup field is estimated to begin producing oil in 2019 and so is not present in the data set used for the analysis.}  

Exploration in the Norwegian Sea was opened in the early 1980’s and the first commercial field started production in 1981.  While several mid-sized fields have been discovered, the Norwegian Sea has generally disappointed in terms of commercial oil finds and most finds have been relatively small.  

Norwegian waters in the Barents Sea have also been open to exploration since the 1980s, however up until recently only a few, small finds were made and none came into commercial production.  Recently several significant oil and gas finds have been made in the Barents Sea - notably the oil field Goliat and the large gas find Sn\o hvit, which are both currently under development but not yet producing.  The agreement between Norway and Russia in June 2011 settling a long-running dispute over the maritime delimitation has also given a boost to new exploration in the region.  

Profits from oil and gas production in Norway are subject to a resource tax of 51 \% on top of the ordinary income tax of 27 \%, thus income from petroleum production is taxed at a total marginal tax rate of 78 \%.  The central government also receives revenues through ownership stakes in companies, notably Statoil, where the state is the majority stakeholder.  

The over-all tax rate has been fairly constant through the history of Norwegian oil production, however several important changes related to the tax code have occurred.  In 1991 a C02 tax was introduced, and in the year 2012 the tax was doubled.  But this was mainly levied on the use and import of petroleum products.  In the offshore sector it was levied on the burning of oil and gas and thus the main effect was on the practice of flaring natural gas that could not be transported and sold commercially.

More important to the offshore sector were accounting changes that were implemented in 2002 and 2005, which were meant to encourage new entrants by allowing losses to be carried forward for tax purposes and by introducing a rebate on the tax-value of losses associated with searching and drilling.  In general though, these rules mainly affected searching and discoveries of new fields rather than production from existing fields and I do not control for the tax changes in my model.  More information on taxation and revenues from the offshore sector can be found at the website of the Norwegian Ministry of Finance. \footnote{\url{http://www.regjeringen.no/en/dep/fin/Selected-topics/taxes-and-duties/bedriftsbeskatning/Taxation-of-petroleum-activities.html?id=417318}}

Rights to explore and eventually produce on the Norwegian Continental Shelf are based on a system where the government announces geographic blocks that will be opened to oil exploration and production subject to production licenses.  Production licenses are initially granted for between 4-6 years subject to requirements that firms are actively searching in their awarded blocks.  If oil or gas deposits are proven then the production license can be extended for up to 30 years.  In general, the frameworks are fair, predictable and stable for companies who find commercially extractable oil deposits and regulatory interference is unlikely to be the cause of any observed changes in production from existing oil fields.  For more information see the website of the Norwegian Petroleum Directorate. \footnote{\url{http://www.npd.no/en/Topics/Production-licences/Theme-articles/Production-licence--licence-to-explore-discover-and-produce-/}}

\section{Data}
Production data of Norwegian oil fields is obtained from the website of the Norwegian Petroleum Directorate.\footnote{http://factpages.npd.no/} The data I use includes all current or formerly oil-producing fields, from 1971 through 2014. Production data is available at a monthly frequency, though I choose to aggregate up to yearly values both to smooth over seasonality as well as short-term volatility of output due to factors such as weather or technical issues that are not relevant for this article. 

In addition to data on field-level production, I also have access to data on estimates of total recoverable reserves as well as estimates of total in-place oil for each field. The use of total recoverable reserves is complicated since it is dependent on the price of oil and thus endogenous in the analysis.  Therefor, as a proxy for the size of the fields, I use estimates of in-place oil from the Norwegian Petroleum Directorate, which is unlikely to be correlated with prices. The left panels of figure \ref{data_descriptives} show the histogram of field sizes in my data set. 

\begin{figure}
	\includegraphics[width=1\textwidth]{figures/data_descriptives.png}
	\caption{Data descriptives. The left panels show histograms of estimates of recoverable oil and in-place oil by field. The right panels shows the production cost index and real brent oil prices.}
	\label{data_descriptives}
\end{figure}

To simplify the analysis, I eliminate the smallest fields from my analysis - those with less than 3 million Standard Cubic Meters (SM3) of recoverable reserves.  These fields tend to have substantially different production profiles than the larger fields. However, these fields only account for around .5 \% of total oil production so this exclusion is unlikely to have any serious practical implications. 

I also eliminate the large Ekofisk field from my analysis.  I do this because it has a unique double-humped production profile. Ekofisk was the first field to be found and the first to start production on the Norwegian Continental Shelf.  It reached an initial peak in production in the late 1970's and began depleting. However, the introduction of water and gas injection technology led to a boost in production and a new peak in production in the early 2000's. All subsequent major fields used gas and water injection and have a single-hump profile.  

I use yearly data from the US Energy Information Agency on the real price of Brent-traded oil in 2010 dollars. The Brent benchmark oil price is likely the best oil price measure for Norwegian production as it is based on light sweet crude oil sourced from the North Sea.  

Production costs are known to vary substantially with the price of oil.  To control for this, I include the producer price index for Norwegian oil and gas extraction industry, obtained from Statistics Norway. \footnote{\url{https://www.ssb.no/en/priser-og-prisindekser/statistikker/ppi}} This series only goes back to 2000, however, so I also include an index of oil and gas well costs from the US energy information agency for the years 1960 to 2000.  \footnote{\url{http://www.eia.gov/dnav/pet/pet_crd_wellcost_s1_a.htm}} While this measure is unlikely to be a perfect substitute for costs in Norway, oil and gas extraction is a global industry, where costs are highly correlated across regions. The combined production cost index and the real Brent oil price series are shown in the right panels in figure \ref{data_descriptives}.

An argument can be made that expectations of future oil prices can be equally if not even more important for production decisions as the current oil price.  Forecasts for future oil prices are available from, among others, the International Energy Agency, but these have tended to be notoriously inaccurate and it is unlikely oil companies use these projections for their investment decisions.

On the other hand, given the size and liquidity of oil spot markets, it is a fair assumption that the current oil prices do a good job of incorporating much of the available information about crude oil markets and that future price movements are generally difficult to predict \citep{hamilton_understanding_2008}.

An active futures market does exist, but several studies have found that current oil prices are in general better than prices on futures contracts at predicting future oil prices \citep{alquist_what_2010, chinn_predictive_2005}.  \citet{mohn_investment_2008} as well as \citet{pesaran_econometric_1990} and \citet{farzin_impact_2001} find evidence for adaptive expectations, where expectations of future prices is based on a weighted average of current and past prices.  I take account of this by including several years of price lags in my regression equations.  

A cleaned data set and the full code for the analysis are available at \url{jmaurit.github.io/#oil_prices}.

\section{A generalized additive model of oil field production}
Parametric linear models have the sizeable advantages of simplicity and interpretability and are therefore usually a good starting point for an analysis. However, when attempting to model the effect of price on oil field production, a standard linear model is unable to sufficiently control for the production profile and therefore heavily biases the estimate of the effect of price.
  
Instead of attempting to estimate the shape of the production profiles of the fields by estimating parameters on linear terms I estimate a non-parametric function for the production profile. At a conceptual level, this model is decomposing the variation in the data into a smoothed component representing the technical and physical forces that produce the typical bell-shaped production profile, and the remaining component, that I estimate the effect of price with.  

I do not, however, want to estimate smoothed curves individually for each field. While this would provide a good overall fit to the full data set, not enough variation in the data would be left to estimate the effect of price.  Instead I want to estimate a general shape of the production profile for all fields and then use the remaining variation in the data to estimate the effect of price.  

A simple model can be written as in equation \ref{gam_price_eqn}. 

\begin{equation}
\begin{split}

	Log(production_{i,t}) & = f(production\_time_{i,t}) + \beta_1 in\_place\_oil_{i,t} \\
	 \quad & + \beta_2 oil\_price\_lags_t + \beta_3 cost\_index_{t} +  
	 f(year_t) + \epsilon_{i,t}
\label{gam_price_eqn}
\end{split}
\end{equation}

In this model I am estimating the parameters and functions from all fields $i$. The left-hand-side variable, $Log(production_{i,t})$ is the log of yearly oil production for field $i$ at production time $t$ in millions of standard cubic meters (SM3).  On the right hand side is a smoothed function of the production profile over time $f(production\_time_{i,t})$; $in\_place\_oil_{i,t}$ is the estimated in place oil of the field, and $cost\_index_{t}$ represents the producer price index for oil and gas extraction. 

$f(year_t)$ represents a function of the calendar year.  The main purpose of including this term is to control for the effects of technological change.

Finally, $oil\_price\_lags_t$ represents the variables of interest - a vector of concurrent and lagged terms for the oil price. Initially I included eight lagged terms of the oil price, however in the preferred specification I drop lags 5 through 8 as they where not individually significant, nor as a group did they improve the fit of the model - as tested by an F-test. I also experimented with using quadratic terms for the oil price and its lags, however these could not be shown to be significant. 

The smoothed function is estimated using a cubic regression spline. Here a cubic polynomial function is used to fit the shape in sections, separated at points known as knots, but continuous up to the second derivative.  The major advantage of this method, is that it can be represented in linear form $\boldsymbol{X \beta} $, and thus normal matrix algebra techniques can be used.  

The model is then fit by minimizing,

\begin{equation}

||\boldsymbol{y} - \boldsymbol{X\beta} ||^2 + \lambda \int_{0}^{1} [f´´(x)]^2dx

\end{equation}

The latter term, an estimate of the second derivative of the function, acts a penalty for the ``wiggiliness'' of the function. The total smoothness can then be controlled by adjusting the $\lambda$ term. However, instead of setting this arbitrarily, cross-validation is used. Intuitively, each data point is left out and the smooth term that provides the best predicted fit for each data point is chosen.  For further details on the cubic regression spline, I refer to \citet{wood_generalized_2006}.

The advantage of this simple model that use smoothed functions of single-variable cubic regression splines is that the smoothed terms can be interpreted directly, and thus we can do a visual check of their appropriateness. The top panel of figure \ref{smooths} shows the estimated smoothed function of the field production profile, where all other co-variates are held at their average value.  The function appears reasonable, showing the expected pattern of an initial build-out period followed by an extended depletion phase. 

In the second panel, the smoothed function of calendar time $year_t$ is shown. Here a cubic regression spline is also used, however the number of knots is restricted to four in order to avoid over-fitting - capturing gradual trends over time like technological change, but not the variation  caused by volatile oil prices. The function increases strongly in the early period, likely reflecting the effects of learning and technological change in a new industry in Norwegian waters, while flattening out in the latter years. 

Extended differences in the oil price will likely have some long-term effects on technology and learning, which in turn may be picked up in the smoothed function here, rather than in the parametric price terms.  I discuss this further in the conclusion.  

\begin{figure}
	\includegraphics[width=1\textwidth]{figures/smooths.png}
	\caption{The estimated smoothed functions over production time and calendar time.}
	\label{smooths}
\end{figure}

One objection to the simple model is that a fixed effect controlling for the general production profile of oil fields does not sufficiently address idiosyncratic differences between the fields. To address this I can add field random effects that takes into account uncertainty based on random variation between the fields in the estimation. The smoothed function can then be written as $f(production\_time_{i,t})*Z$.  Where $Z~N(0, \sigma)$ and $\sigma$ is the variance term to be estimated.  

Another potential issue may be that the smoothed function of production time may not be sufficient in controlling for the variation in production profiles shapes across fields.  In particular, production profiles differ substantially by the size of the field. A solution is to allow the shape to vary by production.  Thus the smoothed can be written as a function of both production time and in place oil: $f(production\_time_{i,t}, in\_place\_oil_{i,t})$.

Estimating a smoothed function of two variables requires a somewhat more involved method.  In particular, a Thin-plate regression spline \citep{wood_thin_2003} is used. Consider equation \ref{thin_plate_1}. 

	\begin{equation}
	y_i = g(x_1, x_2)
	\label{thin_plate_1}
	\end{equation}

Following the notation of \citet{wood_generalized_2006}, $g$ is the function of $x_1$ and $x_2$ that is to be estimated by $f$, which in turn is estimated by minimizing equation \ref{thin_plate_2}.  Here $\boldsymbol{y}$ represents a vector of $y_i$’s and $\boldsymbol{f} = (f(\boldsymbol{x_1}),f(\boldsymbol{x_2}))^t$.   

	\begin{equation}
\min \|\boldsymbol{y-f}\|^2 + \lambda J_{22}(f)
\label{thin_plate_2}
	\end{equation}

$J_{22}$ represents the penalty function for the smoothness of the function which can be written as in \ref{thin_plate_3}.  The $22$ represents the fact that it is a penalty function of two variables with smoothness measured by the second derivative.

	\begin{equation}
	J_{22}{f}= \diffp[2]{f}{x_1}^2 + \diffp[2]{f}{{x_1}{x_2}} + \diffp[2]{f}{x_2}^2dx_1 dx_2
\label{thin_plate_3}
	\end{equation}

In short, a function of $x_1$ and $x_2$ is found that is minimizing errors in the sense of minimizing Euclidean distance subject to a penalty function of “wiggiliness”.  The actual implementation is somewhat more involved in order to increase the computational efficiency of the estimation.  For further details I again refer to \citet{wood_thin_2003}. 

Unfortunately, estimating a random effects component is not possible with a two-dimensional smoothed function for computational reasons.  In the presentation of results, I therefor present models side-by-side that use both one-dimensional smooth functions with random effects of the field production profile and two-dimensional smooth functions.

\section{The Effect of Oil Price on Field Production}

Table \ref{GAM_model_table} shows the parametric coefficient terms and other summary statistics of the regression results.  The first column shows results for the simple model with a single-dimensional smoothed function of the production profile.  The second column shows results with field-level random effects, while the third columns shows results when a two-dimensional smoothed function is used to control for production profiles across field size. The latter two models are my preferred models, providing a substantially better fit, as measured by both deviance explained and R^2. Monte Carlo Experiments, which I describe further below, also indicates that the latter two models provide consistently less-biased parametric estimates of the price terms.  

The variables of interest in this article is the oil price and its lags, which I include as eight linear parametric terms in the model.  The idea of including both a concurrent oil price term as well as eight lags is that a change in price could conceivably have two effects on oil production in a field.  First, the field operator could be operating on the basis of some short-term extraction rule - choosing to pump out less at times of lower prices in order to pump out more at periods of high prices.  

Alternatively, a change in price of oil can be seen as a lifting of a production constraint.  A higher oil price means that added investments in production become attractive in order to either increase the total amount extracted from a field or to shift production forward.  However investments in the offshore sector can be complex and lengthy, and any production lift would be expected to happen with a lag.  As mentioned earlier, including several lags also allows for the possibility of adaptive expectations of future oil prices.  

\begin{table}
\begin{center}
\begin{tabular}{l c c c }
\hline
                                             & w/out Rand. Effects & w Rand. Effects & 2-d Smooth \\
\hline
(Intercept)                                  & $0.335^{*}$    & $-4.017^{***}$ & $-574.287^{***}$ \\
                                             & $(0.146)$      & $(0.183)$      & $(51.125)$       \\
I(in\_place\_oil\_mill\_sm3/100)             & $0.337^{***}$  & $-0.014$       &                  \\
                                             & $(0.005)$      & $(0.014)$      &                  \\
I(cost\_index/10)                            & $-0.018^{***}$ & $-0.030^{***}$ & $-0.014^{***}$   \\
                                             & $(0.003)$      & $(0.003)$      & $(0.002)$        \\
I(price/10)                                  & $-0.006$       & $0.014$        & $0.007$          \\
                                             & $(0.009)$      & $(0.010)$      & $(0.006)$        \\
I(price\_l1/10)                              & $0.004$        & $0.021^{*}$    & $0.015^{*}$      \\
                                             & $(0.010)$      & $(0.010)$      & $(0.006)$        \\
I(price\_l2/10)                              & $0.037^{***}$  & $0.047^{***}$  & $0.048^{***}$    \\
                                             & $(0.009)$      & $(0.010)$      & $(0.006)$        \\
I(price\_l3/10)                              & $0.030^{***}$  & $0.025^{**}$   & $0.033^{***}$    \\
                                             & $(0.009)$      & $(0.009)$      & $(0.006)$        \\
I(price\_l4/10)                              & $0.003$        & $0.043^{***}$  & $0.028^{***}$    \\
                                             & $(0.008)$      & $(0.008)$      & $(0.005)$        \\
I(price\_l5/10)                              & $-0.020^{**}$  & $0.012$        & $-0.001$         \\
                                             & $(0.007)$      & $(0.007)$      & $(0.005)$        \\
I(price\_l6/10)                              & $-0.010$       & $0.008$        & $-0.001$         \\
                                             & $(0.007)$      & $(0.007)$      & $(0.004)$        \\
I(price\_l7/10)                              & $-0.008$       & $0.007$        & $0.004$          \\
                                             & $(0.007)$      & $(0.007)$      & $(0.004)$        \\
I(price\_l8/10)                              & $-0.044^{***}$ & $-0.011$       & $-0.027^{***}$   \\
                                             & $(0.006)$      & $(0.007)$      & $(0.004)$        \\
EDF: s(prod\_year)                           & $6.381^{***}$  &                &                  \\
                                             & $(7.249)$      &                &                  \\
EDF: s(year)                                 & $2.768^{***}$  & $3.000^{***}$  & $3.000^{***}$    \\
                                             & $(2.956)$      & $(3.000)$      & $(3.000)$        \\
EDF: s(prod\_year,name)                      &                & $33.002^{***}$ &                  \\
                                             &                & $(72.000)$     &                  \\
EDF: s(prod\_year,in\_place\_oil\_mill\_sm3) &                &                & $27.445^{***}$   \\
                                             &                &                & $(27.895)$       \\
\hline
Deviance                                     & 23062.678      & 19765.833      & 9057.204         \\
Deviance explained                           & 0.848          & 0.870          & 0.940            \\
Dispersion                                   & 5.410          & 5.965          & 2.325            \\
R$^2$                                        & 0.841          & 0.865          & 0.938            \\
GCV score                                    & 18.145         & 16.218         & 7.355            \\
Num. obs.                                    & 1088           & 1088           & 1088             \\
Num. smooth terms                            & 2              & 2              & 2                \\
\hline
\multicolumn{4}{l}{\scriptsize{$^{***}p<0.001$, $^{**}p<0.01$, $^*p<0.05$}}
\end{tabular}
\caption{Model results. Standard errors in parenthesis.}
\label{GAM_model_table}
\end{center}
\end{table}
 
For convenience the estimated coefficients on the price terms from the two preferred models are shown in figure \ref{price_coefficients}. In general the results from the two models agree with each other, indicating little to no effect in the concurrent price term, but with significant effects estimated on the 1st to 4th lags, with a peak at the 2nd lag.  F-tests of restricted models without the concurrent and 5-8th lag confirm that these terms do not significantly improve the fit of the models.

\begin{figure}
	\includegraphics[width=1\textwidth]{figures/price_coefficents.png}
	\caption{Estimated coefficients from two GAM models: a model with random field effects and single smoothed function of production time, and another with a two dimensional smoothed function of production time and field size.  The points represent the point estimates, while the lines represent approximate 95\% confidence intervals. Significant positive coefficients are estimated on lags of between 1-4.  However, little to no effect is measured on the concurrent price term.}
	\label{price_coefficients}
\end{figure}

For computationally intensive Generalized Additive Models, few if any analytic results are available to confirm that the estimates are unbiased.  Instead, I rely on Monte Carlo experiments to test for biased results.  Here, I create artificial field data using a cumulative logistic function with a linear price term and random variation. I then estimate the effect of price repeatedly, generating a new random component before each new estimation. Details of the Monte Carlo experiment can be found in the online appendix \url{https://github.com/jmaurit/oil_prices/blob/master/mc_sim.html}.  

The main results are shown in figure \ref{mc_results}.  The top panel shows one run of the artificially generated data. I compare estimations of a concurrent price term, $beta_hat$ made with a Generalized Linear Model and Generalized Additive Model with a 2-dimensional smooth function.  I set the true parameter $beta$ to both $0$ and $0.05$, which are shown as vertical lines in the figures.  The figure shows that while a bias exists in the estimation of the GLM model, especially for when the true value is $0.05$, the Generalized Additive Model generally provides unbiased results. 

\begin{figure}
	\includegraphics[width=1\textwidth]{figures/mc_plot.png}
	\caption{Results from a Monte-Carlo experiment. I generate data mimicking oil price production from a set of fields, as displayed in the top frame. Using GLM and GAM models, and regenerating the error component in the data, I replicate estimates of the effect of price on production, which is set at $\beta = 0$ and $\beta=0.05$. The density of the results for the GLM model are shown in the lower left panel, where a clear and large bias is shown relative to the true $\beta$ which are represented by the vertical lines.  A GAM model however provides estimates close to the true $\beta$}
	\label{mc_results}
\end{figure}

\section{Conclusion}

The main results of this research is to show that production in existing Norwegian offshore fields has no significant concurrent reaction to higher oil prices while a slight effect is estimated  with a lag of between 1 and 4 years.  Oil producers do not appear to be behaving strategically in relation to short-term production - increasing or reducing production in response to changes in oil price.  Instead they are likely using storage or financial instruments to hedge short-term price movements. Changes in oil prices can rather be seen as a relaxing of a production constraint, justifying increased investment that leads to either a higher total extraction rate or an inter-temporal shifting of production.

The modest estimated effect of prices on production adds weight to the argument of \citet{hamilton_oil_2012} that most of the increased supply of oil that comes from higher prices is from expanding the geographic and technological boundaries of oil production.  For example exploration of deep-water oil deposits off the coast of Brazil and extraction of oil sands in western Canada.   

\section{Software}
I use the R statistical programming package for all the analysis in this article \citep{r_core_team_r:_2013}.  I use the R package ggplot2 for plotting \citep{wickham_ggplot2:_2009}, plyr for data manipulation and cleaning \citep{wickham_split-apply-combine_2011}, texreg for table formatting \citep{leifeld_texreg:_2013} and mgcv for implementation of the Generalized Additive Models \citep{wood_fast_2011}.

\bibliographystyle{plainnat}
\bibliography{oil_prices}

\FloatBarrier
\end{document}